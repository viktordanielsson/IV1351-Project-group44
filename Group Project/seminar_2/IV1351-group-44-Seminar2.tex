\documentclass[a4paper]{scrartcl}
\usepackage[utf8]{inputenc}
\usepackage[english]{babel}
\usepackage{graphicx}
\usepackage{lastpage}
\usepackage{pgf}
\usepackage{wrapfig}
\usepackage{fancyvrb}
\usepackage{fancyhdr}
\usepackage{hyperref}
\usepackage{minted}
\usemintedstyle{manni}
\usepackage{svg}
\pagestyle{fancy}

% Create header and footer
\headheight 27pt
\pagestyle{fancyplain}
\lhead{\footnotesize{Data Storage Paradigms, IV1351}}
\chead{\footnotesize{Seminar 2 Report}}
\rhead{}
\lfoot{}
\cfoot{\thepage\ (\pageref{LastPage})}
\rfoot{}

\title{Seminar 2 Report}
\subtitle{Data Storage Paradigms, IV1351}
\author{Viktor Danielsson, Armin Eghtesadi}
\date{\today}

\begin{document}

\maketitle
\noindent\textbf{Project members:} \\ \hfill
[Viktor Danielsson, vdani@kth.se] \\ \hfill
[Armin Eghtesadi, armineg@kth.se] \\ \hfill

\section*{Declaration:}

By submitting this assignment, it is hereby declared that all group members listed above have contributed to the solution. It is also declared that all project members fully understand all parts of the final solution and can explain it upon request.

It is furthermore declared that the solution below is a contribution by the project members only, and specifically that no part of the solution has been copied from any other source (except for lecture slides at the course IV1351), no part of the solution has been provided by someone not listed as a project member above, and no part of the solution has been generated by a system.

\section{Introduction}
Queries are the operations which allows users to gather information stored in the database. A customer would use the queries to interface with the database according to the operations defined in the requirement specification. This task focuses on the defined operations and how to get their functionally in SQL language.
\begin{enumerate}
    \item OP1: Calculate the total planned hours for the current years course instance
    \item OP2: Calculate the hours allocated for each teacher within a single course instance
    \item OP3: Calculate the total hours a teacher has allocated for each of their courses
    \item OP4: Display the number of courses instances each teacher is allocated to during the current period.
\end{enumerate}
Then a formal analysis of the performance of some opertations have to be performed to measure how well each query is able to perform the wanted tasks. Query analysis is done by the EXPLAIN ANALYZE postgres command.
\bigskip
\\
Higher grade criteria is met when the tasks have been evaluted based on their frequency and if they should be further optimsed. The frequencies are the following for each task:
\begin{enumerate}
    \item  Planned hours per course instance (8x/day)
    \item Actual allocated hours per teacher per course instance (12x/day)
    \item Teacher load per period (5x/day) 
    \item Course instances with planned vs actual variance  $>15\%$ (3x/day)
    \item Teachers allocated to more than N courses in current period (20x/day)
\end{enumerate}
They are then evalutated based on whether they should be given indices and/or turned into materialized views. This is considered based on their respective query frequency, expected data size, freshness etc. Lastly the improvments will be analyzed and compared against their previous implementation to see performance gains and reason around the benefits/downsides of the new implementation.
\section{Literature Study}
For this seminar sql is needed to understand how the queries are formulated. For this chapters 3-5 were read in foundations of database systems and the lecture on SQL was read. Lastly, the tips and tricks for seminar 2 was used to further understand some concepts.
\section{Method}
Writing a SQL query based on a written task is done by understanding what data is needed to fulfill the task. By joining tables together and filtering its content or by performing artihmetic operations the queries can be combined into a operation much more complex. The analysis of a given query will be performed by EXPLAIN ANALYZe which returns the time to perform the query. 
\bigskip
\\
Evaluating the need for indices and materialized views is based on how complex and often the  query has to be performed. Where a slow, complex and often used query should be optimsed with indices for speed and a materialized view to ease the use of the query. 
\section{Result}
A query for calculating the total planned hours for the current years course instances is written by first joining all relevant tables together. In this scenario tables \_\_\_\_\_\_ where needed and where joined together with \_\_\_\_\_\_. Then by filtering by the current year and summing together the hours the final total hours column can be correctly written.
\bigskip
\\
The second query is done by again analysing which tables are needed to perform the given task. Then joining all the tables togheter and filtering their content such that only relevant data remains. 
\bigskip
\\
A teacher is found in the employee table and their allocated activites in the Employee\_load\_allocation table. Then by
\section{Discussion}

\section{Comments About the Course}

\end{document}
