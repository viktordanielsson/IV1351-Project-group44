\documentclass[a4paper]{scrartcl}
\usepackage[utf8]{inputenc}
\usepackage[english]{babel}
\usepackage{graphicx}
\usepackage{lastpage}
\usepackage{pgf}
\usepackage{wrapfig}
\usepackage{fancyvrb}
\usepackage{fancyhdr}
\usepackage{hyperref}
\usepackage{minted}
\usemintedstyle{manni}
\usepackage{svg}
\pagestyle{fancy}

% Create header and footer
\headheight 27pt
\pagestyle{fancyplain}
\lhead{\footnotesize{Data Storage Paradigms, IV1351}}
\chead{\footnotesize{Seminar 1 Report}}
\rhead{}
\lfoot{}
\cfoot{\thepage\ (\pageref{LastPage})}
\rfoot{}

\title{Seminar 1 Report}
\subtitle{Data Storage Paradigms, IV1351}
\author{Viktor Danielsson, Armin Eghtesadi}
\date{2025-11-14}

\begin{document}

\maketitle
\noindent\textbf{Project members:} \\ \hfill
[Viktor Danielsson, vdani@kth.se] \\ \hfill
[Armin Eghtesadi, armineg@kth.se] \\ \hfill

\section*{Declaration:}

By submitting this assignment, it is hereby declared that all group members listed above have contributed to the solution. It is also declared that all project members fully understand all parts of the final solution and can explain it upon request.

It is furthermore declared that the solution below is a contribution by the project members only, and specifically that no part of the solution has been copied from any other source (except for lecture slides at the course IV1351), no part of the solution has been provided by someone not listed as a project member above, and no part of the solution has been generated by a system.

\section*{Tips for Report Writing}
\textbf{REMOVE THIS SECTION BEFORE SUBMITTING THE REPORT.}\\

\noindent \textit{The target audience has exactly the same skills as the author, except they do not know anything at all about the specific application described in the report.} \\

Consider the following:

\begin{itemize}
  \item \textbf{The report must be \textit{centered around the requirements}. Which are they (Introduction), how did you work to meet them (Method), what is the solution that meets them (Result), and how can you be sure they are met (Discussion). This is the IMRaD method.} The requirements on the Introduction, Method, Result and Discussion chapters are described below under each chapter.

  \item Is spelling and grammar correct? Is spoken language avoided?

  \item Does the report have a good structure with sections, subsections and paragraphs?

  \item Is the text clarified with images and/or other figures, and with links to the code in your Git repository? Remember that all figures (images, tables, graphs, code listings, etc) shall be numbered and have a short explaining text.
\end{itemize}

\section{Introduction}

\textbf{This chapter tells \textit{what} are you going to do.} 

Explain the task and the requirements on the solution. It's important to clearly state the requirements. \textit{Also specify which other student you worked with when solving the tasks, or if you worked alone.} 

The task at hand is to design and create a database capable of describing a universities course system. There has to exist a course layout that describes the course code, course name, how many points the course is worth, the minimum and maximum amount of students. A course instance is needed to track the total amount of students registered and in which period the course is given in for example P1, P2, P3, P4. Each course has many different teaching activites such as lectures, labs, tutorials and seminars. Every teaching activity requires a diffrerent amount of time for the teacher to prepare (multiplication factor) which has to be included in the database. Every course has a varying amount of time for each teaching activity which also has to be considered in the database. The last things are the different departments with their manager (who is also an employee) and the teachers/employees themselves. An employee or teacher works at a department and must have contact details, salary, teaching activites based on skill set, job title and their supervisor/manager. Each teacher can be involved in many teaching activites for multiple different course instances and each course instance can have multiple teachers involved at once. However, a teacher can only be in up to four course instances in a single period.
\bigskip
\\
The final database has to be able to handle all the affermentioned details from the university and be flexible enough to add more  teaching activites later on. It must also be possible to add different course versions of the same course layout to to have a 15hp course aswell as a 7.5hp for the same course.
\bigskip
\\
A university customer would have to specify which operations the database must handle but for a general solution the operations are defined as follows:
\begin{itemize}
  \item A teacher is not allowed to be overloaded with work during a period.
  \item Since costs are computed for all activities given during the particular period, the database must contain information about who gave which activity for which course instance.
  \item It must be possible to add more teaching activities.
  \item It must be possible to change salary for teachers.
  \item A course must be able to be planned based on the hours needed for teaching activities.
  \item When a teaching activity has been made a teacher must be able to be allocated to said activity.
\end{itemize} 
\section{Literature Study}

This chapter must prove that you collected sufficient knowledge before starting development, instead of just hacking away without knowing how to complete a task. State what you have read and briefly summarize what you have learned.
\bigskip
\\
Some of the knowledge gathered to complete the task was: 
\begin{itemize}
    \item Relational data model: What we are making is a relational database built on the relational data model. It stores its data in a collection of tables all of which contain columns to describe what the table contains. The rows describe an instance of the table. The relationship between tables describes how they are related and what type of cardinality they have.
    \item Cardinality: A relationship between two tables must express how many of such tables can be associated with the other table. One-to-one would indicate that the their only exists a single table related to another single table. However Many-to-many would instead indicate multiple tables can have a single relationship to multiple other tables. Cardinality is also necessary to define how many of a certain attribute is allowed in a signle instance, can we have a null amount, one to many, zero to many and so on.
    \item Attribute Domain: For each column or attribute there has to be an assigned domain to define all the values that would be allowed to fill that column for each instance of the table. 
    \item Atomic values: Our database strives to use atomic values in our columns. That means that each value is in its smallest part and not able to be divided into any more smaller parts. So our domain can be atomic if it only allows for a single integer in its column.
    \item Normalization: A database should not store loads of redundant data, it is space inefficient and requires more computation when you perform operations on the database. So normalization is design decisition that formally defines how normalized the data is in the database. The first normal form 1NF is when every single attribute is atomic. NF2 is the state of the model where every attribute is fully dependent on the entire primary key and 3NF has removed all the transitive functional dependencies in a lossless manner (no information was lost).
    \item Primary Key: A primary key is a specific attribute choosen specifically to identify each tuple in the relation, or alternativly each row in a table. 
    \item Foreign key: The foreign key is a value in a different table that has to match one of the values in the primary key of that other table. This allows for references between tables. 
\end{itemize}
\section{Method}

\textbf{This chapter tells \textit{how} you solved the task.}

Explain how you worked when solving the tasks and how you evaluated that your solution met the requirements. \textit{Do not explain your solution and do not refer to code}, that belongs to the \textit{Result} chapter. More specific instructions for the content can be found under each task on the Project page in Canvas.
\bigskip
\\
The method of creating a database starts by creating a miniworld of reality that describes how different entities interact and exchange information Then the conceptual model is transformed into a logical and physical model which describes how the miniworld would be represented in a database. The eleven steps necessary to formulate the correct logical and physical model were:
\begin{description}
  \item Two trivial steps of creating tables and rows based on the conceptual model.
  \item Then assigning the domain for each attribute.
  \item "UNIQUE" attributes will be marked in the model.
  \item Define primary keys for each table which is definied as a strong entity in the model. Typically surrogate key.
  \item Then define the relationships between tables, one-to-many, one-to-one, many-to-many
  \item Generate pk and fk for multivalued attribute tables
  \item Normalize the model
  \item Final step of making the model is verifying all the operations.
\end{description}
After each step is completed the model is complete and the tool used to make the model Astah professional is able to generate a rough SQL dump. Finalizing the database is done by adding UNIQUE to the correct attributes and adding foreign key constraints ON DELETE.
\bigskip
\\
When the schema has been defined a process of adding data has to be added and then some SQL operations can be performed on the model to see that all the data that should be accesable is. 
\section{Result}
Generating a simple logical model with just tables and attributes directly from the conceptual model generated the basis for the finished logical model.
A finalized version of the entire logical model can be seen in figure \ref{fig:image_of_logical_physical_model}
\begin{figure}[h!]
  \begin{center}
    \includesvg[width=\textwidth]{LogPhyModel_image.svg}
    \caption{A complete image of the logical and physical model}
    \label{fig:image_of_logical_physical_model}
  \end{center}
\end{figure}
\\
Assiging the domain to each attribute is a necessary step for the database to allow enough space for each column but also to apply the corresponding constraints which are associated within a domain. In the case of not knowing what the customer wants some liberty has to be taken because it is simply not possible to know exactly what domain a customer would want. 
The domain for each column in person is VARCHAR 13 for the phone number allow a phone number with country code that is larger than 2 digits. Adress, last\_name and first\_name all use VARCHAR 200 to allow for large names but without risks of massive values. personal\_number is unique because two people are not allowed to have the same personal number. 
\bigskip
\\
Attributes for employee: employment\_id is unqiue to an employee and gets the VARCHAR 200 domain. What format the university would use is not known so a variety of formats is allowed by the database. Then an employee has a manager which is also an employee which is why it is allowed to be null. max\_allowed\_courses is an int, salary is an int. job\_title of an employee is a foreign key to the table containing all job\_titles. 
\bigskip
\\
Attributes for course\_layout: We can not not the corse\_code so we allow for 200 charachter VARCHAR so the university can use whatever format it chooses. Course\_name gets the same treatment.
\bigskip
\\
attribute for planned\_acitivty: int for hours
\bigskip
\\
An employee has a manager\_id which is also a employee\_id and for the department it is also a employee\_id which is the managers\_id.  
\bigskip
\\
Each attribute such as person\_number, employment\_id and all the others marked with UNIQUE  got labels describing that property.
\bigskip
\\
Each attribute was chosen to be atomic and also relevant to how a university would use that value. most values used as an id is for business logic for the university and is not to be confused by the surrogate id which identifies the tuples. 
\bigskip
\\
Finding the strong entities was done by looking at the conceptual model and searching for an entity which is not dependant on another entity to exist. Such entites were found to be person, course layout and department. They were then given a surrogate key to define their primary key. 
\bigskip
\\
The SQL dump was generate by Astah by exporting the E-R diagram. This allowed us to quickly deploy a revised database and test it almost immediatly. So if the new design was able to solve issues or not  we could figure it out much faster than if we had to write a new dump by ourselves each time. The SQL dump is published on \href{https://github.com/viktordanielsson/IV1351-Project-group44/blob/main/Group%20Project/university_db_dump3.sql}{github} but the code structure is similar to this for each table 
\begin{figure}[h!]
  \begin{center}
    \begin{minted}[
]{postgres}
CREATE TABLE employee (
 id INT GENERATED ALWAYS AS IDENTITY NOT NULL PRIMARY KEY,
 employment_id  VARCHAR(200) NOT NULL UNIQUE,
 salary INT NOT NULL,
 person_id  INT NOT NULL UNIQUE,
 department_id INT NOT NULL,
 max_allowed_courses INT NOT NULL,
 job_title_id INT NOT NULL,
 manager_id VARCHAR(200),

 FOREIGN KEY (person_id ) 
    REFERENCES person (id)
    ON DELETE CASCADE,
 FOREIGN KEY (department_id) 
    REFERENCES department (id)
    ON DELETE NO ACTION,
 FOREIGN KEY (job_title_id) 
    REFERENCES job_title (id)
    ON DELETE NO ACTION
);
\end{minted}
  \end{center}
\end{figure}

\textbf{This chapter explains \textit{the result} of what you did.}

\textbf{The report must show that you have done the work yourself and that you have understood what you have done}, both of these goals are met by carefully explaining your solution here in the result chapter, and proving that it meets the requirements. \textit{State each requirement that is met} and explain \textit{how you met it}. Also include links to your code in your Git repository, and include also diagrams, see Figure , and other figures to illustrate your reasoning. All figures must be referenced in the text. Ask yourself if the solution is clearly explained, and if the reader will understand the application. What would you yourself want to know if you read about the application, is that included in the report? More specific instructions for the content can be found under each task on the Project page in Canvas. 

Foreign key constraint
\begin{figure}[h!]
  \begin{center}
  \end{center}
\end{figure}

\pagebreak

\section{Discussion}
In general each step has been revised over multiple iterations because the actions following a step might show flaws in the previous steps. One such flaw which showed up only after trying to add data was foreign keys which had "circular" relationsships. Creating one table would first need a different table to be generated first but for that to happen the other table needs the current table, a catch-22 for generating data. These flaws were difficult to find directly in the model but obivous when the constraint violations are thrown in the databsase. These issues were prevelant in both the courses, employees with department but solved by having fewer NOT NULL constraints and less identifying relations. 
\bigskip
\\
The employee attribute skill set is tricky because multiple employees can share the same skill and we want to ensure the values are atomic. So multiple employees can have the same skill and a single employee has multiple skills. The many-to-many relation is solved with a cross-reference table.
\bigskip
\\
For our many-to-many relations such as course instance and teaching activity and employee skillset we use the cross-reference table. It could be done with less tables such as tables with multivalued attributes but it is valuable to decrease the amount of duplicate data
\bigskip
\\
When finding strong entites the department entity is perculiar because the manager\_id is dependent on a employee existing. By allowing for a null manager we can define the department as a strong entity because it is no longer dependant on the employee table. This is allowed because null is not accounted for when checking the column for uniqueness. This was also a point where "circular" dependencies showed up when generating data and having the department independent from employee allows us to create the department without an employee.
\bigskip
\\
When considering the hp for a course there could be the same course with the same course id but with a different amount of hp. Our solution is to divide course and course layout in to two tables. The course only contains code and name with a surrogate key. Then layout has min and max students, hp, a version and a surrogate key as primary key. The relation between the two is non-identifying but one could argue that the layout should be identified by its course. Another way of solving it would be to combine the two tables and have the version as a primary key togehter with course\_id but that solution was not used because the course\_code and course\_name are not related to version in the primary key. 


\textbf{This chapter \textit{analysis} the result presented in the previous section.} 

Evaluate your solution according to the assessment criteria found in the assessment-criteria documents, which are found under the bullet \textit{In the Discussion chapter of your report...}, under each task on the Project page in Canvas. You do not have to cover all specified criteria.


\section{Comments About the Course}
The time invested to the project were six, five hour sessions so in total 30 hours to complete the project.  

\end{document}
